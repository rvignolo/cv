\cvsection{Professional Experience}

\begin{cventries}

  \cventry
    {CRISIL}
    {Lead Quantitative Analyst or Technical Manager}
    {Buenos Aires, Argentina}
    {Apr. 2019 - Present}
    {
      \begin{cvitems}
        \item{Leading technical design, development and implementation of a a
        high-performance library for fast and advanced quantitative finance calculations,
        including:
          \vspace{+6.0mm}
          \begin{cvitems}
              \item Numerical Solution of Stochastic Differential Equations using time
              discrete approximations with different strong/weak order of convergence,
              non-autonomous coefficients and w/ and w/o diagonal noise, which allow solving
              any System of SDEs found in the financial domain.
              \item One and Multi-Factor Short Rate Models of Affine and Quadratic type,
              including efficient on the fly computation of Zero Coupon Bonds by solving the
              corresponding Riccati System of Ordinary Differential Equations. These Short
              Rate Model types take into account the simple and well known cases: Vasicek,
              Cox-Ingersoll-Ross, Hull-White, Gaussian Short Rate (GSR) and Quadratic
              Gaussian. However, many more complicated models can be specified and solved as
              well.
              \item Libor Market Model (LMM) framework including all basic fixed income
              securities and interpolation using, for example, the Schlögl methodology.
              \item Heath–Jarrow–Morton (HJM) general framework including all basic fixed
              income securities.
              \item Local Volatility and Stochastic Volatility models.
              \item Monte Carlo universal pricing engine for non-callable, callable and
              cancellable equity and hybrids products with arbitrary basis layers (from
              polynomials to neural networks). Greeks computation via automatic
              differentiation.
              \item Day counters for many different conventions.
              \item Yield curve construction.
              \item Domain Specific Language (DSL) design and implementation for
              syntactically-sweetened inputs.
              \item Automatic documentation for both code and theory.
          \end{cvitems}
          \vspace{+4.0mm}
        % por si quiero agregar algo aqui abajo o un proximo item, va el +4 arriba
        }
        % \item{}
      \end{cvitems}
      \vspace{1em}
    }

  \cventry
    {CRISIL}
    {Senior Quantitative Analyst}
    {New York, USA -- Buenos Aires, ARG}
    {Sep. 2017 - Mar. 2019}
    {
      \begin{cvitems}
        \item{Consultant for Tier-1 US Investment Bank – Model Validation Group:
          \vspace{+6.0mm}
          \begin{cvitems}
              \item {Development of plausible forecasts for macroeconomic and market
              variables in the context of Current Expected Credit Loss (CECL) models by
              coding a forecasting suite which implemented:
                \vspace{+6.0mm}
                \begin{cvitems}
                  \item $\textsf{AR}(p)$, $\textsf{MA}(q)$, $\textsf{ARMA}$,
                  $\textsf{ARIMA}$ and Seasonal $\textsf{ARIMA}$ + Independent Variables
                  ($\textsf{SARIMAX}$) models.
                  \item Error Correction Model ($\textsf{ECM}$).
                  \item Kalman Filtering.
                  \item Automatic documentation including many statistical tests presented
                  in both textual and graphical format, allowing quick identification of
                  model correctness.
                \end{cvitems}
                \vspace{+4.0mm}
              }
          \end{cvitems}
          \vspace{+8.0mm}
        }
        \item{Consultant for Tier-1 US Investment Bank – Front Office – Equity \& Hybrids Group:
          \vspace{+6.0mm}
          \begin{cvitems}
            \item {Worked on different performance testing strategies and model life-cycle
            duties for a variety of Products and Base models, including:
              \vspace{+6.0mm}
              \begin{cvitems}
                \item Analyzing different underlying dynamics for currencies, equity, FX and
                interest rate processes using several types of Volatility Models (Local
                Volatility and Stochastic Volatility) and Term Structure Models
                (deterministic rates, Short Rate Models of Affine and Quadratic class and
                Libor Market Model) for benchmarking purposes.
                \item Pricing by means of Analytic, Trees, Finite Difference (PDE) and Monte
                Carlo methods.
                \item Benchmarking against different Product Models by formulation of
                comprehensive comparisons.
                \item Parametric testing modifying relevant dynamics and/or payoff related
                parameters.
                \item Life-cycle testing for schedule sensitive parameters.
                \item Limiting cases validation collapsing each product model to simple
                Vanilla-Like Derivatives, among others.
                \item Calibration impact studies for each Model Dynamics using different
                methodologies.
                \item Multi-currency curve handling and construction rationale.
                \item Risk-Not-In-Model assessment against complex dynamics, e.g. Stochastic
                Volatility dynamics w/ and w/o jumps.
                \item Convergence, computational performance and Hedging studies.
                \item Stress Testing scenarios for standardized \& required regulatory
                scenarios.
                \item Technical documentation, where all the relevant information and
                results were detailed for the correct and comprehensive use of each Product
                Model.
              \end{cvitems}
              \vspace{+4.0mm}
          }
          \end{cvitems}
          \vspace{+4.0mm}
        }
      \end{cvitems}
      \vspace{1em}
    }

  \cventry
    {BESNA}
    {Nuclear Engineering Consultant}
    {Buenos Aires, Argentina}
    {Dec. 2016 - Dec. 2017}
    {
      \begin{cvitems}
        \item{Design and development of Heat Transfer – Two-Phase Flow complex calculation
        codes for Nuclear Reactors:
          \vspace{+6.0mm}
          \begin{cvitems}
            \item Implemented a program that performs steady-state analysis of
            thermo-hydraulic systems involving mixtures of steam and water (Two-Phase Flow)
            in one dimensional, but still complex, geometries. Coupled capabilities with
            other CFD tools (Ansys CFX) was included as well.
            \item Calculations for the Helical-Coiled Steam Generator of CAREM25 Nuclear
            Reactor.
          \end{cvitems}
          \vspace{+4.0mm}
        }
      \end{cvitems}
      \vspace{1em}
    }

  \cventry
    {TECNA}
    {S/Sr Nuclear Engineer}
    {Buenos Aires, Argentina}
    {Oct. 2014 - Aug. 2017}
    {
    \begin{cvitems}
        \item{Development of Reactor Physics (neutronic, thermo-hydraulic and control) high
        performance calculation codes:
            \vspace{+6.0mm}
            \begin{cvitems}
                \item Implemented the Method of Characteristics approximation for the
                Neutron Transport Equation (including efficient ray tracing algorithms).
                \item Designed and developed cell models to obtain condensed and homogenized
                macroscopic cross sections with DRAGON V5 for Atucha II Nuclear Power Plant.
                \item Consultancy regarding the development of computational codes in order
                to reproduce coupled safety transients and update the Final Safety Analysis
                Report (FSAR) for Atucha I Nuclear Power Plant.
                \item Consultancy regarding Spatial Kinetics calculations coupled with plant
                and control codes.
                \item Conducted several updates for Neutron Spatial Kinetics programs.
            \end{cvitems}
            \vspace{+4.0mm}
        }
    \end{cvitems}
    \vspace{1em}
    }

\end{cventries}
